\documentclass[letterpaper,12pt]{article}

%%%%%%%%%%%%%%%%%%%%%%%%%%%%%%%%%%%%%%%%%%%%%%%%%%%%%%%%%%%%%%%%%%%%%%%%%%%%%%
%%% TWEAK THE FOLLOWING BASED ON WHAT YOU THINK THE CURRENT DOCUMENT NEEDS %%%
%%%%%%%%%%%%%%%%%%%%%%%%%%%%%%%%%%%%%%%%%%%%%%%%%%%%%%%%%%%%%%%%%%%%%%%%%%%%%%

\usepackage[inner=1.5cm,outer=1.5cm,top=1.8cm,bottom=1.5cm]{geometry}
\usepackage[colorlinks=true,linkcolor=blue,urlcolor=blue]{hyperref}
\DeclareMathSizes{12pt}{12pt}{10pt}{8pt}

%%%%%%%%%%%%%%%%%%%%%%%%%%%%%%%%%%%%%%%%%%%%%%%%%%%%
%%%%%%%%%  IMPORT TEMPLATE FILES AS NEEDED %%%%%%%%
%%%%%%%%%%%%%%%%%%%%%%%%%%%%%%%%%%%%%%%%%%%%%%%%%%%%

\usepackage{amsgen,amsmath,amstext,amsbsy,amsopn,amssymb,amsthm}
\usepackage[usenames,dvipsnames,svgnames,table]{xcolor}
\usepackage{array, nicefrac, mathtools}
\usepackage[bottom]{footmisc} % To keep the footnote at the bottom even after putting answering environments.
\usepackage{verbatim}
\usepackage{booktabs} % for multicolumn
\usepackage[font=normalsize,skip=0pt, justification=centering]{caption, subcaption}
\usepackage[colorlinks=true,linkcolor=blue,urlcolor=blue]{hyperref}
\usepackage{float,relsize,setspace,enumitem,pbox,cleveref,multicol}
\usepackage{censor}
\usepackage{textcomp} % for the cent symbol
\usepackage{multido}
\usepackage{bbding} % Has a checkmark symbol reachable through \Checkmark
\usepackage{tikz}
\newcommand{\myline}[1]{\underline{\hspace{#1}}}
\newcounter{problems}
\newcounter{questions}[problems]
\newcounter{subquestions}[questions]
\newcommand{\problem}[2]{\stepcounter{problems} {\Large \bf \noindent Problem \arabic{problems}: #1 \emph{(#2 pts)}} \\[.4cm]}
\newcommand{\question}[2]{\stepcounter{questions} \noindent{\emph{\Large Question (\alph{questions}): #1 (#2 pts) }\\[.4cm]}}
\newcommand{\subquestion}[2]{\stepcounter{subquestions} \noindent{\emph{\Large Subquestion (\roman{subquestions}): #1 (#2 pts) }\\[.4cm]}}

% Some standard centering and italicization of text.
\newcommand{\frontrowcenter}[1]{\begin{center}{\em \Large  #1  }\end{center}}

% A blank page
\newcommand{\blankpage}{
\clearpage
\vspace*{\fill}
	\begin{minipage}{\textwidth}
		\hspace{.5in} \Large \textbf{THIS PAGE INTENTIONALLY LEFT BLANK}\\
	\end{minipage}
\vfill % equivalent to \vspace{\fill}
\clearpage
}

\newcommand{\answerspace}[1]{
	\begin{center}
		\textbf{BEGIN YOUR ANSWER BELOW THIS LINE} \\ \hrulefill \vspace{#1} \\ \hrulefill
	\end{center}
}

\newcommand{\answerspacefullpage}{
	\begin{center}
		\textbf{BEGIN YOUR ANSWER BELOW THIS LINE} \\ \hrulefill \pagebreak
	\end{center}
}


\newcommand{\additionalanswerspace}[1]{
	\begin{center}
		\textbf{CONTINUE YOUR ANSWER FOR #1 BELOW THIS LINE } \\ \hrulefill \vspace{#1} \\ \hrulefill
	\end{center}
}

\newcommand{\additionalanswerspacefullpage}{
	\begin{center}
		\textbf{CONTINUE YOUR ANSWER BELOW THIS LINE} \\ \hrulefill \pagebreak
	\end{center}
}

\newcommand{\freespace}[1]{
	\begin{center}
		\large \textbf{SCRAP SPACE BELOW} \\ 
		\hrulefill
		\pagebreak 
	\end{center}
}

\newcommand{\silentanswerspace}[1]{
	\\ \hrule \vspace{#1} \hrule 
}


\newcommand{\notespage}{
	\pagebreak
	\begin{center}
		\Large \textbf{SCRAP PAPER} \\ \hrulefill
	\end{center}
	\pagebreak
}

\newcommand{\showyourwork}{
	\begin{center}
		\large \textbf{SHOW YOUR WORK BELOW} \\ 
		\hrulefill
		\pagebreak
	\end{center}
}

\newcommand{\justifywork}{
	\begin{center}
		\large \textbf{JUSTIFY YOUR ANSWERS BELOW} \\ 
		\hrulefill
		\pagebreak
	\end{center}
}

% Space for T/F:
\newcommand{\tfline}{\myline{.5cm}}

% For sets:
\newcommand{\curlybraces}[1]{\ensuremath{\lbrace #1 \rbrace}}

% For cardinalities:
\newcommand{\crd}[1]{\ensuremath{ \big \vert #1 \big \vert }}

\newcommand{\yesno}{{\footnotesize YES / NO}}
\newcommand{\truefalse}{{\normalsize  \bf TRUE / FALSE}}

% \item environments coupled with a line at the end, for students to write T and F in.
\newcommand{\tfitem}[1]{\item #1 \null\hfill \framebox(25,25){} \\ \hdashrule{0.95\textwidth}{1pt}{2pt}}
\newcommand{\setitem}[1]{\tfitem{$\curlybraces{#1}$} }
\newcommand{\lineitem}[1]{\item #1 \null \hfill \myline{1in}}

% Some circles for people to fill in,
\newcommand{\whitecircle}[1]{\tikz[baseline=-0.5ex]\draw[black, radius=#1] (0,0) circle ;}
%\newcommand{\blackcircle}[2][black,fill=black]{\tikz[baseline=-0.5ex]\draw[black,radius=#1] (0,0)}
\newcommand{\blackcircle}[1]{\tikz\draw[black,fill=black,radius=#1] (0,0) circle (.5ex);}%
% Basic math
\usepackage{amsgen,amsmath,amstext,amsbsy,amsopn,amssymb,amsthm}
\usepackage[usenames,dvipsnames,svgnames,table]{xcolor}
\usepackage{array, nicefrac, mathtools}

% Theorems, definitions, equations, lemmas
\newtheorem{thm}{Theorem}[section]
\newtheorem{prop}[thm]{Proposition}
\newtheorem{lem}[thm]{Lemma}
\newtheorem{cor}[thm]{Corollary}
\newtheorem{defn}{Definition}
\newtheorem{rem}[thm]{Remark}
\numberwithin{equation}{section}
\newtheorem*{defn*}{Definition} % Theorem environments with no numbering
\newtheorem*{prop*}{Proposition}
\newtheorem*{thm*}{Theorem}

% For negation and quantifiers in Discrete Math
\newcommand{\shortsim}{\raise.17ex\hbox{$\scriptstyle \sim$}}
\renewcommand{\neg}{\shortsim}
\renewcommand{\nexists}{\neg\exists}
\newcommand{\nequiv}{\ensuremath{\not\equiv}}

% Some larger symbols for clarity.
\newcommand{\Sum}{\ensuremath{\mathlarger{\sum}}}
\newcommand{\Prod}{\ensuremath{\mathlarger{\prod}}}
\newcommand{\Ell}{\ensuremath{\mathcal{L}}}
\DeclarePairedDelimiter{\ceil}{\lceil}{\rceil}
\DeclarePairedDelimiter{\floor}{\lfloor}{\rfloor}

%Some number sets
\newcommand{\N}{\ensuremath{\mathbb{N}}}
\newcommand{\Nplus}{\ensuremath{\mathbb{N}^+}}
\newcommand{\Nstar}{\ensuremath{\mathbb{N}^*}} % pretty much equivalent to Nplus
\newcommand{\Neven}{\ensuremath{\mathbb{N}^\text{even}}}
\newcommand{\Nstareven}{\ensuremath{\mathbb{N}^*_\text{even}}}
\newcommand{\Nodd}{\ensuremath{\mathbb{N}^\text{odd}}}
\newcommand{\Z}{\ensuremath{\mathbb{Z}}}
\newcommand{\Zstar}{\ensuremath{\mathbb{Z}^*}}
\newcommand{\Zstareven}{\ensuremath{\mathbb{Z}^*_\text{even}}}
\newcommand{\Zplus}{\ensuremath{\mathbb{Z}^+}}
\newcommand{\Zminus}{\ensuremath{\mathbb{Z}^-}}
\newcommand{\Zeven}{\ensuremath{\mathbb{Z}^\text{even}}}
\newcommand{\Zodd}{\ensuremath{\mathbb{Z}^\text{odd}}}
\newcommand{\Q}{\ensuremath{\mathbb{Q}}}
\newcommand{\Qplus}{\ensuremath{\mathbb{Q}^+}}
\newcommand{\Qstar}{\ensuremath{\mathbb{Q}^*}}
\newcommand{\Qminus}{\ensuremath{\mathbb{Q}^-}}
\newcommand{\R}{\ensuremath{\mathbb{R}}}
\newcommand{\Rminus}{\ensuremath{\mathbb{R}^-}}
\newcommand{\Rplus}{\ensuremath{\mathbb{R}^+}}
\newcommand{\Rstar}{\ensuremath{\mathbb{R}^*}}
\newcommand{\I}{\ensuremath{\mathbb{R - Q}}}
\renewcommand{\P}{\ensuremath{\mathbf{P}}}
\newcommand{\Pset}[1]{\ensuremath{\mathcal{P}(#1)}}

% An explained mathematical derivation in the form of an unbulleted list item.
\newcommand{\derivitem}[2]{\item[] $=#1 \qquad \qquad $ \derivexpl{#2}}
\newcommand{\derivitemnte}[2]{\item[] $#1 \qquad \qquad $ \derivexpl{#2}}
\newcommand{\derivitemized}[2]{\item $=#1 \qquad \qquad $ \derivexpl{#2}}
\newcommand{\derivitemizednte}[2]{\item $#1 \qquad \qquad $ \derivexpl{#2}}

% Math and lines.
\newcommand{\mathitem}[1]{\item $#1$ \hrulefill}

% Induction-related
\newcommand{\indstart}[1]{Let $P($#1$)$ be the proposition we are attempting to prove true. We proceed via mathematical induction on #1.}
\newcommand{\IB}{\textbf{Inductive Base: }}
\newcommand{\IH}{\textbf{Inductive Hypothesis: }}
\newcommand{\IS}{\textbf{Inductive Step: }}
\newcommand{\derivexpl}[1]{\text{ \emph { (#1) } } \\ } % Textual explanation of line-by-line derivations

% Interesting mathematical notation and symbols

\newcommand{\bigoh}[1]{$\mathcal O$(#1)}
\newcommand{\bigtheta}[1]{$\mathit{\Theta}$(#1)}
\newcommand{\bigomega}[1]{$\mathit{\Omega}$(#1)}

% For logical rules of inference

\newcommand{\rulesofinference}[2]{
	\begin{table}[H]
		\centering
		\begin{tabular}{|c|c|p{2.5cm}|p{2.5cm}|}\hline
			\centering
			\textbf{Modus Ponens} & \textbf{Modus Tollens} & \textbf{Disjunctive addition} & \textbf{Conjunctive addition} \\ \hline
				$\begin{aligned}
			p  \\
			p \Rightarrow q \\
			\therefore q
		\end{aligned}$ & 
		 $\begin{aligned}
			\neg q  \\
			p \Rightarrow q \\
			\therefore \neg p
		\end{aligned}$ & 
		 $\begin{aligned}
			p  \\
			\therefore p \lor q
		\end{aligned}$ &
		$\begin{aligned}
			p, q \\
			\therefore p \land q
		\end{aligned}$ \\ \hline 
			 \textbf{Conjunctive Simplification} & \textbf{Disjunctive syllogism} & \textbf{Hypothetical syllogism} &  \textbf{Resolution}  \\ \hline
			 $\begin{aligned}
			p \land q \\
			\therefore p, q
		\end{aligned}$ 	& 
			$\begin{aligned}
			p \lor q \\
			\neg p \\
			\therefore q
		\end{aligned}$ &
		$\begin{aligned}
			p  \Rightarrow q \\
			q \Rightarrow r \\
			\therefore p \Rightarrow r
		\end{aligned}$ &  
		$\begin{aligned}
		p \lor q\\
		(\neg q) \lor z\\
		\therefore p \lor z
		\end{aligned}$ \\ \hline
		\end{tabular}
		\caption{#1}
		\label{#2}
	\end{table}
}

\newcommand{\logicalequivs}[2]{

	\begin{table}[H]
		\centering
		\renewcommand*{\arraystretch}{1.2}
		\begin{tabular}{|>{\centering\arraybackslash}p{2.5in} | c | c |} \hline
			\textbf{Commutativity of binary operators} & $p \land q \equiv q \land p$ & $p \lor q \equiv q \lor p$ \\ \hline
		\textbf{Associativity of binary operators} & $(p \land q) \land r \equiv p \land (q \land r)$ &  $(p \lor q) \lor r \equiv p \lor (q \lor r)$ \\ \hline
		\textbf{Distributivity of binary operators} & $p \land (q \lor r) \equiv (p \land q) \lor (p \land r)$ & $p \lor (q \land r) \equiv (p \lor q) \land (p \lor r)$ \\ \hline
		\textbf{Identity laws} & $p \land T \equiv p$ & $p \lor F \equiv p$ \\ \hline
		\textbf{Negation laws} & $p \lor (\neg p) \equiv T$ & $p \land (\neg p) \equiv F$ \\ \hline
		\textbf{Double negation} & \multicolumn{2}{c|}{$\neg (\neg p) \equiv p$}  \\ \hline 
		\textbf{Idempotence} & $p \land p \equiv p$ & $p \lor p \equiv p$ \\ \hline
		\textbf{De Morgan's axioms} & $\neg (p \land q) \equiv (\neg p )\lor (\neg q)$ & $\neg (p \lor q) \equiv (\neg p) \land (\neg q)$\\ \hline
		\textbf{Universal bound laws} & $p \lor T \equiv T$ & $p \land F \equiv F$ \\ \hline
		\textbf{Absorption laws} & $p \lor (p \land q) \equiv p$ & $p \land (p \lor q) \equiv p$ \\ \hline
		\textbf{Negations of contradictions / tautologies} & $\neg F \equiv T$ & $\neg T \equiv F$ \\ \hline
		\textbf{Contrapositive} & \multicolumn{2}{c|}{$(a \Rightarrow b) \equiv (( \neg b) \Rightarrow (\neg a))$} \\ \hline
		\textbf{Equivalence between biconditional and implication} & \multicolumn{2}{c|}{$a \Leftrightarrow b \equiv (a \Rightarrow b) \land (b \Rightarrow a)$} \\ \hline
		\textbf{Equivalence between implication and disjunction} & \multicolumn{2}{c|}{$a \Rightarrow b \equiv \neg a \lor b$} \\ \hline
		\end{tabular} \vspace{.1in}
		\caption{#1}
		\label{#2}
	\end{table}
}

\newcommand{\settheory}[2]{
	\begin{table}[H]
		\centering
		\renewcommand*{\arraystretch}{1.4}
		\begin{tabular}{| c | c | c | } \hline 
			{\large \bf Operation} & 		{\large \bf Symbol} &  {\large \bf Definition } \\  \hline 
			\textbf{Membership} & $x \in A$ & $x$ is a member of set $A$ \\ \hline
			\textbf{Non-membership} & $x \notin A$ & $ \neg (x \in A)$ \\ \hline
			\textbf{Union} & $A \cup B$ & $\{ (x \in A) \lor (x \in B)$\}   \\ \hline
			\textbf{Intersection} & $A \cap B$ & $\{ (x \in A) \land (x \in B)$\}   \\ \hline
			\textbf{Relative complement of} $\mathbf B$ \textbf{given} $\mathbf A$ & $A - B$ & $\{ (x \in A) \land (x \notin B)$\}   \\ \hline 
			\textbf{Universal (Absolute) complement} & $\overline{A}$ & $\{x \notin A\}$\\ \hline
			\textbf{Cartesian Product} & $A \times B$ & $\{(a, b) \mid   (a \in A) \land (b \in B)  \}$ \\ \hline
			\textbf{Subset} & $A \subseteq B$ & $(\forall x \in A)[x \in B] $\\ \hline
			\textbf{Superset} & $A \supseteq B$ & $ B \subseteq A$\\ \hline
			\textbf{Set equality} & $A = B$ & $(A \subseteq B) \land (B \subseteq A) $ \\ \hline
			\textbf{Set non-equality} & $A \neq  B$ & $\neg (A =B) $ \\ \hline
			\textbf{Proper subset} & $A \subset B$ & $ \{ (A \subseteq B) \land (A \neq B) \} $ \\ \hline
			\textbf{Proper superset} & $A \supset B$ & $ \{ (A \supseteq B) \land (A \neq B) \} $ \\ \hline
			\textbf{Powerset} & $\Pset{A} $ & $ \{ X \mid X \subseteq A \} $ \\ \hline
		\end{tabular}
		\vspace{.1in}
		\caption{#1}
		\label{#2}
	\end{table}
}

% A useful environment for the questions where I ask students to provide two infinite domains
% which make a quantified statement true and false. Needs \myline.

\newcommand{\innerlist}{
	\begin{itemize}
		\setlength\itemsep{1em}	
		\item[-] $D_T=$ \myline{2in}
		\item[-] $D_F=$ \myline{2in}
	\end{itemize}
}

% Induction - related
\newcommand{\indbase}[1]{
	\begin{center}
		\textbf{WRITE YOUR INDUCTIVE BASE BELOW THIS LINE} \\
		\hrulefill
	\end{center}
   	\vspace{#1}
}

\newcommand{\indhypothesis}[1]{
	\begin{center}
		\textbf{WRITE YOUR INDUCTIVE HYPOTHESIS BELOW THIS LINE} \\
		\hrulefill
	\end{center}
   	\vspace{#1}
}


\newcommand{\indstep}[1]{
	\begin{center}
		\textbf{WRITE YOUR INDUCTIVE STEP BELOW THIS LINE} \\
		\hrulefill
	\end{center}
   	\vspace{#1}
}


\newcommand{\contindbase}[1]{
	\begin{center}
		\textbf{CONTINUE YOUR INDUCTIVE BASE BELOW THIS LINE} \\
		\hrulefill
	\end{center}
   	\vspace{#1}
}

\newcommand{\contindhypothesis}[1]{
	\begin{center}
		\textbf{CONTINUE YOUR INDUCTIVE HYPOTHESIS BELOW THIS LINE} \\
		\hrulefill
	\end{center}
   	\vspace{#1}
}


\newcommand{\contindstep}[1]{
	\begin{center}
		\textbf{CONTINUE YOUR INDUCTIVE STEP BELOW THIS LINE} \\
		\hrulefill
	\end{center}
   	\vspace{#1}
}


\newcommand{\standardinductionspace}{
	\indbase{2.5in}
	\indhypothesis{1.5in}
	\pagebreak
	\indstep{\paperheight}
}
% Emphasis
\newcommand{\F}{$\mathbf{F}$}
\newcommand{\T}{$\mathbf{T}$}
\newcommand{\False}{\textbf{False}}
\newcommand{\false}{\textbf{false}}
\newcommand{\True}{\textbf{True}}
\newcommand{\true}{\textbf{true}}
\newcommand{\TODO}{\textcolor{red}{\textbf{TODO}}}
\newcommand{\TBD}{\textcolor{red}{\textbf{TBD}}}
\newcommand{\codeemph}[1]{\texttt{\textbf{#1}}}
\newcommand{\Red}{\textcolor{red}{Red}}
\newcommand{\red}{\textcolor{red}{red}}
\newcommand{\makered}[1]{\textcolor{red}{#1}}
\newcommand{\Rbbst}{\textcolor{red}{Red}-black tree}
\newcommand{\rbbst}{\textcolor{red}{red}-black tree}
\usepackage{censor} 
\usepackage{textcomp} % for the cent symbol
\usepackage{multido}
\newcommand{\nullc}{\texttt{\textbackslash 0}}
\newcommand{\SPC}{\texttt{SPC}}
\usepackage{bbding} % Has a checkmark symbol reachable through \Checkmark
\newcommand{\checkmarkifsoln}{\ifshowsoln \textcolor{red}{\Checkmark} \fi}
\newcommand{\yesno}{{\footnotesize YES / NO}}
\newcommand{\truefalse}{{\normalsize  \bf TRUE / FALSE}}
\newcommand{\dialitem}[2]{\item[-] \textbf{#1}: ``\textit{#2}"} % For dialogue building.
% Repeating commands many times!
\newcommand{\Repeat}{\multido{\i=1+1}} 
% % Source code, mainly for Data Structures
\definecolor{DarkGray}{gray}{0.35}
\definecolor{dkgreen}{rgb}{0, 1, 0}
\definecolor{mauve}{rgb}{78, 9,117}
\usepackage{listings, textcomp, upquote}
\lstset{
  language=Java,
  upquote=true;
  aboveskip=3mm,
  belowskip=3mm,
  showstringspaces=false,
  columns=flexible,
  numbers=left,
  basicstyle={\small\ttfamily},
  numberstyle=\tiny\color{black},
  keywordstyle=\color{blue},
  commentstyle=\color{DarkGray},
  stringstyle=\color{ForestGreen},
  breaklines=true,
  breakatwhitespace=true,
  tabsize=3
}
%\newcommand{\homeworkdata}[3]{

		\noindent\makebox[\textwidth]{\LARGE \bf #1, #2 }
		\ \\
		\noindent\makebox[\textwidth]{\Large \bf  Homework \##3 }
	
		\ \\
		% STUDENT: If you want to change your first and last name, all you have to do is replace \myline{4in} with \underline{John Doe}.
			\noindent\makebox[\textwidth]{\large \bf  \textbf{\underline{First}} \& \textbf{\underline{Last}} Name:  \enspace \myline{4in}}
		 \Repeat{2}{\ \\ }
			\noindent\makebox[\textwidth]{\large \bf  UID (9 digits): \enspace \myline{3in}}
		\ \\ 
}
\newcommand{\examdata}[5]{
   \newpage
   \setcounter{page}{1}
   \noindent

   \begin{center}
      \hspace*{-.7cm}
	   \framebox{
	      \vbox{
	        \vspace{2mm}
		    \hbox to 7.3in { {\bf #1 \hfill Sections: #2} }
	    	\vspace{8mm}
	        \hbox to 6.9in { {\huge  \hfill  #3 \hfill } }
	        \vspace{4mm}
     	        \hbox to 6.9in { {\Large  \hfill   \textbf{Date}: #4 \hfill }} 
     	        \vspace{4mm}
     	        \hbox to 6.9in { {\Large  \hfill   \textbf{Time}: #5 \hfill } }  	             	       
     	         
       	        \vspace{8mm}
   \hbox to 7.2in { {\it First and last name ({\bf exactly} as on ELMS): \myline{3in} \hfill} }
	        \vspace{4mm}
   \hbox to 7.7in {  {\it UID (9 digits):} \myline{1.4in} \hfill	
   
}				
			
	        \begin{center}
	        	\textbf{University Honor Pledge:} \\[.2in]
	        	\emph{I pledge on my honor that I have not given or received \\
		        any unauthorized assistance on this assignment/examination.  }\\[.35in]		        
				\textbf{Print the text of the University Honor Pledge below}: \\[.2in]
				 \myline{5in} \\[.1in]
				\myline{5in} \\[.1in]
				\myline{5in} \\[.3in]
			\textbf{Signature:} \myline{2.8in}  \\
			\end{center}
	   	  } % vbox
	   } % framebox
   \end{center}
   \markboth{#3-- #1}{#3 -- #1}
   \vspace*{4mm}
} 

%%%%%%%%%%%%%%%%%%%%%%%%%%%%%%%%%%%%%%%%%%%%%%%%%%
%%%%%%%%%%%%%%%%%%%  END OF IMPORTS %%%%%%%%%%%%%%%
%%%%%%%%%%%%%%%%%%%%%%%%%%%%%%%%%%%%%%%%%%%%%%%%%%

%%%%%%%%%%%%%%%%%%%%%%%%%%%%%%%%%%%%%%%%%%%%%%%%
%%%%%%%%%  COMMAND (RE)DEFINITION AREA  %%%%%%%%
%%%%%%%%%%%%%%%%%%%%%%%%%%%%%%%%%%%%%%%%%%%%%%%%

% If a template command doesn't work the way we want it to for the
% current document, a sound strategy would be to redefine it here
% using \renewcommand. Can also use this area for our own custom
% commands that we want effective on the current document.

%%%%%%%%%%%%%%%%%%%%%%%%%%%%%%%%%%%%%%%%%%%%%%%%%%
%%%% END OF COMMAND (RE)DEFINITION AREA  %%%%%%%%%
%%%%%%%%%%%%%%%%%%%%%%%%%%%%%%%%%%%%%%%%%%%%%%%%%%

\title{CMSC250, Fall '18: Midterm \#1}

\begin{document}

\examdata{CMSC 250, Fall 2018}{020x}{Midterm Exam\#1}{Wednesday, October 10th}{06:0\textbf{5}-08:00pm}

\vspace{-.3in}

\section*{Exam guidelines / rules} 

\label{sec:guidelines}
\begin{itemize}
	\item \textbf{Turn off all unapproved electronic devices}.
	\item The exam is {\bf closed book} and {\bf notes}.
	\item There are \textbf{7 (seven)}  problems in this exam, with a total grade value equal to $\mathbf{100}$ \textbf{(one hundred)} points.
	\item The exam  has been printed \textbf{two-sided}, \textbf{stapled on the top-left corner} and spans \textbf{20 (twenty)} pages across \textbf{10 (ten)} sheets. {\bf DO NOT RIP PAGES FROM THE EXAM}. Last page is {\bf scrap paper}. If you need \textbf{extra} scrap paper, ask us for it.
	\item The total time allocated for this exam is \textbf{115 (one hundred and fifteen)} minutes.
	\item You may \textbf{not} leave the classroom (e.g to go to the bathroom, or because you're done) during the \textbf{last 5 (five) minutes} of the exam.
\end{itemize}

\section*{Provided materials \& assumed facts}
\label{sec:materials}

\subsection*{Logic}
\label{subsec:logic}

{\large Table \ref{tbl:propLogicAxioms} contains a number of logical equivalences that we have discussed in class.

\logicalequivs{A number of propositional logic axioms you can refer to.}{tbl:propLogicAxioms}

Table \ref{tbl:validInference} contains a number of {\bf valid} rules of inference / natural deduction that we have discussed in class.

\rulesofinference{Some valid rules of inference / natural deduction you can refer to.}{tbl:validInference}


\subsection*{Set Theory}
\label{subsec:setTheory}

{\large The following are Set Theoretic notation and definitions.

\settheory{Definitions of Set Theory}{tbl:setTheory} }

\subsection*{Others}
\label{subsec:otherAssumptions}

{\large You may assume/use the following mathematical and  logical facts/definitions {\bf without proof}. 

 \begin{itemize}
       \item $\N = \{0,1,2,...\}$ (that is, $0 \in \N$)
		\item $\N$ is closed under \textbf{addition} and \textbf{multiplication}.
		\item $\Z$ is closed under \textbf{addition}, \textbf{subtraction} and 
		\textbf{multiplication}.
		\item $\R$ is closed under \textbf{addition}, \textbf{subtraction} and 
			\textbf{multiplication}.
		\item An integer can be \textbf{either} odd \textbf{or} even, but \textbf{not both}.
\end{itemize}
}
\pagebreak

\problem{Logic}{10}

{\large Suppose that we have the following rule of inference: }

{\large 
\begin{align*} 
		(\neg  p) \Rightarrow (\neg  q) \\ 
		(\neg  p) \lor r \\ 
		\neg  r \\
		\cline{1-2}
		\therefore q
\end{align*} 
}
\question{Truth Table}{3}
			
{\large Fill in the following truth table. Write {\bf neatly}; if we can't make out the difference between \T{} and \F{}, we will be forced to take off points! Also, you {\bf must} write \T{} and \F{}, {\bf not} \textbf{1} or \textbf{0}. }

\newcounter{ttableRows}
\newcommand{\currInd}{\stepcounter{ttableRows}\arabic{ttableRows}}

\begin{center}
	\begin{table}[H] 
		\centering
		\Large 
		\setlength{\tabcolsep}{12pt}
%		\renewcommand{\arraystretch}{1.2}
		\begin{tabular}{|c|c|c|c|c|c|c|c|c|} \hline 
			{\bf Row \# } & $p$ & $q$ & $r$ & $\neg p$ & $\neg{q}$ & $\neg  r$ & $ (\neg  p) \lor r$ & $(\neg  p) \Rightarrow (\neg  q)$ \\ \hline
			\currInd & \F & \F & \F & & &  & & \\ \hline 		
			\currInd & \F & \F & \T & & &  & &\\ \hline 		
			\currInd & \F & \T & \F & & &   & &\\ \hline 		
			\currInd & \F & \T & \T & & & & &\\ \hline
			\currInd & \T & \F & \F & & & & &\\ \hline 		
			\currInd & \T & \F & \T & & & & &\\ \hline
			\currInd & \T & \T & \F & & & & &\\ \hline 		
			\currInd & \T & \T & \T & & & & &\\ \hline												
		\end{tabular}
	\end{table}
\end{center} \vspace{-.3in} 


\question{Rule invalidity}{2}

\vspace{-.3in}

{\large \underline{\bf Briefly} explain on the lines below \textbf{why} the truth table you filled tells us that the proof is {\bf invalid}. DO {\bf NOT} MAKE ANY MARKS ON THE TABLE ITSELF! You can refer to a row by its \textbf{number}.

\begin{center}
	\hrulefill  \\ \ \\ 
	\hrulefill
\end{center}

\pagebreak

\question{Deriving a valid conclusion}{5} 

{\large In questions (a) and (b), we proved that $q$  does {\bf not} logically follow from our given premises. Using \textbf{only} valid rules of natural deduction provided for you in table \ref{tbl:validInference}, prove that $\neg  q$ {\bf does} follow from these premises, establishing the following as a {\bf valid} rule of inference. } You {\bf must} label each step of your proof with the name of the  rule of natural deduction that you use!
{\large 
\begin{align*} 
		(\neg  p) \Rightarrow (\neg  q) \\ 
		(\neg  p) \lor r \\ 
		\neg  r \\
		\cline{1-2}
		\therefore (\neg  q)
\end{align*} 
} \vspace{-.4in}
\answerspacefullpage

\problem{Circuits}{10}

{\large Observe the truth table of Table \ref{tbl:ioTable}. }


\begin{table}[H]
	\centering
	\renewcommand{\arraystretch}{1.2}
{\large
	\begin{tabular}{|c|c|c|c|c|} \hline
		\multicolumn{3}{|c|}{Inputs} & Output \\ \hline 
		$p$ & $q$ & $r$ & $z$ \\ \hline
		\F & \F & \F & \T \\ \hline			
		\F & \F & \T & \F \\ \hline
		\F & \T & \F & \F \\ \hline
		\F & \T & \T & \F \\ \hline
		\T & \F & \F & \T \\ \hline
		\T & \F & \T & \T \\ \hline		
		\T & \T & \F & \F \\ \hline
		\T & \T & \T & \F \\ \hline
	\end{tabular}
	
	}
	\caption{A truth table for a formula that maps inputs $p, q, r$ to an output $z$.}
	\label{tbl:ioTable}
\end{table}

\question{Boolean Formula}{5}

{\large Based on the truth table shown to you in Table \ref{tbl:ioTable}, provide us with a boolean (propositional) formula that computes the output $z$ based on the inputs $p, q, r$. Your formula should be in {\em Disjunctive Normal Form} (\textbf{DNF}), i.e it must be a \textbf{disjunction of conjunctions}, like in your homework.}

\answerspacefullpage

\question{Circuit}{5}

{\large Draw a \textbf{circuit} that computes the boolean formula}

{\Large $$ (p \lor (\neg  q)) \land (r \lor q) \land ((\neg  r) \lor p \ )  $$}

You should use \bf only} {\bf AND, OR} and {\bf NOT} gates!

\answerspacefullpage

\problem{Set Theory}{10}

{\large For every one of the following statements, fill in the circle corresponding to the appropriate choice (\True{} or \False{}).  For example, if any given statement is \True, you should fill in the {\bf first} circle, such that \whitecircle{5pt} becomes \tikz\draw[black,fill=black] (0,0) circle (.85ex); . \textbf{PLEASE DO \underline{NOT} USE CHECKMARKS (\Checkmark),  CROSSES ({\large $\mathlarger \displaystyle \times$}), ETC: FILL-IN THE CIRCLES AS INDICATED ABOVE.}  You {\bf do not} need to justify your answers.}

\newcounter{setQuestions}
\renewcommand{\currInd}{\stepcounter{setQuestions}(\alph{setQuestions})}
\newcommand{\defcircle}{\whitecircle{5pt}}

%%%%%%%%%%%%%%%%%%%%%%%%%%%%%%%%%%%%%%%%%%%%%%%%%%%%%%%%%%%%%%%%%%%%%%%%%%%%%%%%%%%%%%%%%%%%%
% STUDENT: For every one of the following statements, replace the first \defcircle with % 
% \blackcircle{5pt} if you believe the statement is True, otherwise replace the 
% second \defcircle with the same macro.  
%%%%%%%%%%%%%%%%%%%%%%%%%%%%%%%%%%%%%%%%%%%%%%%%%%%%%%%%%%%%%%%%%%%%%%%%%%%%%%%%%%%%%%%%%%%%%

\begin{table}[H]
	\renewcommand{\arraystretch}{2}
	\centering
	{\large 
	\begin{tabular}{|c|c|c|c|} \hline 
		& {\bf Statement } & {\bf True} & {\bf False} \\ \hline 
		\currInd & $0 \in \Q$ & \defcircle & \defcircle \\ \hline
		\currInd & $50 \in \Z$ & \defcircle & \defcircle \\ \hline
		\currInd & $\emptyset  \in \emptyset$ & \defcircle & \defcircle \\ \hline
		\currInd & $\emptyset  \subseteq \curlybraces{0}$ & \defcircle & \defcircle \\ \hline
		\currInd & $(A - B) = A \cup B^c $ & \defcircle & \defcircle \\ \hline
		\currInd & $\N \cup \Z = \Q$ & \defcircle & \defcircle \\ \hline
		\currInd & $ \vert \curlybraces{1, 6} - \curlybraces{\curlybraces{1, 6}} \vert = 2$  & \defcircle & \defcircle \\ \hline
		\currInd & $\emptyset \in \Pset{\curlybraces{\emptyset}}$& \defcircle & \defcircle \\ \hline
		\currInd & $\curlybraces{\sqrt2,  \sqrt4} \in \Pset{\R - \Q}$  & \defcircle & \defcircle \\ \hline%

		\currInd & $(\forall x, y \in \Q^{>0})[x^y \in \Q^{>0}]$  & \defcircle & \defcircle \\ \hline
	\end{tabular}
	}
	\caption*{}
\end{table} 

\pagebreak


\problem{Quantifiers}{15}

{\large In the following quantified expressions, push the negation operator ($\neg $) {\bf as far inside the expression as possible}. Your final expression should {\bf not have any instances of $\neg $!} Do {\bf not} concern yourselves with the {\bf truth values} of the statements {\bf or} their negations; just concentrate on the task at hand.} You do {\bf not} need to show us intermediate steps; just the {\bf final expression} is sufficient.

\begin{enumerate}[label=(\alph*)] 
	\item  {\Large $\neg  (\forall n \in \Z)[n^2 \geq 0]$ } \answerspace{1in}
	\item {\Large $\neg  (\forall n_1 \in \Z)(\exists n_2 \in \Z)[(n_1 - n_2) < 0]$ } \answerspace{1in}
	\item {\Large $\neg  (\forall n_1 \in \Z)(\forall n_2 \in \Z)\bigl [(n_1 < n_2) \Rightarrow (\nexists q \in \Q)[n_1 < q < n_2] \bigr ]$ } \answerspacefullpage
\end{enumerate}
\pagebreak

\problem{Direct proofs}{20} 

\question{Rationals}{5}
  
{\large Prove {\bf directly} that, if  $q \in \Q$, then $\frac{q}{3} \in \Q$}.

\answerspacefullpage

\question{Divisiblity}{15}

{\large Suppose $x, y \in \Z$. When divided by $7$, $x$ leaves a remainder of $2$. When divided by 
$14$, $y$ leaves a remainder of $3$. Prove {\bf directly} that $x \cdot y$ leaves a remainder of $6$ when divided by $7$. }
\answerspacefullpage

\problem{Indirect proofs}{30} 

\question{Lemma}{5} 

{\large \noindent Suppose $n$ is an integer. Prove {\bf indirectly} that, if $n^3$ is a multiple of $5$, then so is $n$. }

\answerspacefullpage
\additionalanswerspacefullpage

\question{Euclidean proof I}{10}

{\large Prove {\bf indirectly} that $\sqrt[3]{5} \notin \Q$ using the {\bf Euclidean argument}. }

\answerspacefullpage
\additionalanswerspacefullpage

\question{Proof via UPFT}{15}

{\large Prove {\bf indirectly} that $\sqrt[3]{5} \notin \Q$ using the\textbf{ Unique Prime Factorization  Theorem}. }
\answerspacefullpage
\additionalanswerspacefullpage

\problem{Show me what you got}{5}

{\large Suppose that $n \in \Z$. Using \textbf{any} methodology that you have learned so far in the class, prove that $8n^2 + 5n$ is even {\bf if, and only if}, $n^4 + 6$ is even.  You can use the page in the back if you run out of space.}

\answerspacefullpage
\additionalanswerspacefullpage
\notespage
\end{document}
