% Document class will always be article for the purposes of UMD assignments
\documentclass[letterpaper,10pt]{article}

%%%%%%%%%%%%%  IMPORT MACRO FILES AS NEEDED %%%%%%%%%%%
% Basic math
\usepackage{amsgen,amsmath,amstext,amsbsy,amsopn,amssymb,amsthm}
\usepackage[usenames,dvipsnames,svgnames,table]{xcolor}
\usepackage{array, nicefrac, mathtools}

% Theorems, definitions, equations, lemmas
\newtheorem{thm}{Theorem}[section]
\newtheorem{prop}[thm]{Proposition}
\newtheorem{lem}[thm]{Lemma}
\newtheorem{cor}[thm]{Corollary}
\newtheorem{defn}{Definition}
\newtheorem{rem}[thm]{Remark}
\numberwithin{equation}{section}
\newtheorem*{defn*}{Definition} % Theorem environments with no numbering
\newtheorem*{prop*}{Proposition}
\newtheorem*{thm*}{Theorem}

% For negation and quantifiers in Discrete Math
\newcommand{\shortsim}{\raise.17ex\hbox{$\scriptstyle \sim$}}
\renewcommand{\neg}{\shortsim}
\renewcommand{\nexists}{\neg\exists}
\newcommand{\nequiv}{\ensuremath{\not\equiv}}

% Some larger symbols for clarity.
\newcommand{\Sum}{\ensuremath{\mathlarger{\sum}}}
\newcommand{\Prod}{\ensuremath{\mathlarger{\prod}}}
\newcommand{\Ell}{\ensuremath{\mathcal{L}}}
\DeclarePairedDelimiter{\ceil}{\lceil}{\rceil}
\DeclarePairedDelimiter{\floor}{\lfloor}{\rfloor}

%Some number sets
\newcommand{\N}{\ensuremath{\mathbb{N}}}
\newcommand{\Nplus}{\ensuremath{\mathbb{N}^+}}
\newcommand{\Nstar}{\ensuremath{\mathbb{N}^*}} % pretty much equivalent to Nplus
\newcommand{\Neven}{\ensuremath{\mathbb{N}^\text{even}}}
\newcommand{\Nstareven}{\ensuremath{\mathbb{N}^*_\text{even}}}
\newcommand{\Nodd}{\ensuremath{\mathbb{N}^\text{odd}}}
\newcommand{\Z}{\ensuremath{\mathbb{Z}}}
\newcommand{\Zstar}{\ensuremath{\mathbb{Z}^*}}
\newcommand{\Zstareven}{\ensuremath{\mathbb{Z}^*_\text{even}}}
\newcommand{\Zplus}{\ensuremath{\mathbb{Z}^+}}
\newcommand{\Zminus}{\ensuremath{\mathbb{Z}^-}}
\newcommand{\Zeven}{\ensuremath{\mathbb{Z}^\text{even}}}
\newcommand{\Zodd}{\ensuremath{\mathbb{Z}^\text{odd}}}
\newcommand{\Q}{\ensuremath{\mathbb{Q}}}
\newcommand{\Qplus}{\ensuremath{\mathbb{Q}^+}}
\newcommand{\Qstar}{\ensuremath{\mathbb{Q}^*}}
\newcommand{\Qminus}{\ensuremath{\mathbb{Q}^-}}
\newcommand{\R}{\ensuremath{\mathbb{R}}}
\newcommand{\Rminus}{\ensuremath{\mathbb{R}^-}}
\newcommand{\Rplus}{\ensuremath{\mathbb{R}^+}}
\newcommand{\Rstar}{\ensuremath{\mathbb{R}^*}}
\newcommand{\I}{\ensuremath{\mathbb{R - Q}}}
\renewcommand{\P}{\ensuremath{\mathbf{P}}}

% Induction-related
\newcommand{\indstart}[1]{Let $P($#1$)$ be the proposition we are attempting to prove true. We proceed via mathematical induction on #1.}
\newcommand{\IB}{\textbf{Inductive Base: }}
\newcommand{\IH}{\textbf{Inductive Hypothesis: }}
\newcommand{\IS}{\textbf{Inductive Step: }}
\newcommand{\derivexpl}[1]{\text{ \emph { (#1) } } \\ } % Textual explanation of line-by-line derivations

% An explained mathematical derivation in the form of an unbulleted list item.
\newcommand{\mathitem}[2]{\item[] $#1 \qquad \qquad $ \derivexpl{#2}}

% An inner list shorthand for one of my Discrete Math exam subjects.
\newcommand{\innerlist}{
	\begin{enumerate}[label=(\roman*)]
		\item $D_T=$ \myline{2in}
		\item $D_F=$ \myline{2in}
	\end{enumerate}
}
% Source code, mainly for Data Structures
\definecolor{DarkGray}{gray}{0.35}
\definecolor{dkgreen}{rgb}{0, 1, 0}
\definecolor{mauve}{rgb}{78, 9,117}
\usepackage{listings, textcomp, upquote}
\lstset{
  language=Java,
  upquote=true;
  aboveskip=3mm,
  belowskip=3mm,
  showstringspaces=false,
  columns=flexible,
  numbers=left,
  basicstyle={\small\ttfamily},
  numberstyle=\tiny\color{black},
  keywordstyle=\color{blue},
  commentstyle=\color{DarkGray},
  stringstyle=\color{ForestGreen},
  breaklines=true,
  breakatwhitespace=true,
  tabsize=3
}
\newcommand{\myline}[1]{\underline{\hspace{#1}}}
\newcommand{\problemhdr}[1]{\noindent{\textbf{\Large Problem #1} \\[.2cm]}}
\newcommand{\questionhdr}[1]{ \noindent{\emph{\large Question #1} \\[.2cm]}}

% Some standard centering and italicization of text.
\newcommand{\frontrowcenter}[1]{\begin{center}{\em \Large ( #1 ) }\end{center}}

\newcommand{\afterquestionvskip}{\ \\[0.02em]}

\usepackage[bottom]{footmisc} % To keep the footnote at the bottom even after putting answering environments.
\usepackage{verbatim}
\usepackage{booktabs} % for multicolumn
\usepackage[font=normalsize,skip=0pt, justification=centering]{caption, subcaption}
\usepackage{float,relsize,setspace, fancyhdr,enumitem,pbox,cleveref,multicol}

% A blank page
\newcommand{\blankpage}{
\clearpage
\vspace*{\fill}
	\begin{minipage}{\textwidth}
		\hspace{.5in} \Large \textbf{THIS PAGE INTENTIONALLY LEFT BLANK}\\
	\end{minipage}
\vfill % equivalent to \vspace{\fill}
\clearpage
}

% Space for students' answers
\newcommand{\answerspace}[3]{
		\begin{center}
			\textbf{BEGIN YOUR ANSWER FOR #1 BELOW THIS LINE} \\
	    	\noindent\rule{#2}{0.4pt} \\
	    	
	    	\vspace{#3} 
	    	
	    	\noindent\rule{#2}{0.4pt}
 		 \end{center}
 		 
}

\newcommand{\answerspacefullpage}[1]{
	\afterquestionvskip
	\begin{center}
		\textbf{BEGIN YOUR ANSWER FOR #1 BELOW THIS LINE} \\ 
		\hrulefill
		
		\pagebreak 
	\end{center}
}

\newcommand{\additionalanswerspace}[1]{
	\afterquestionvskip
	\begin{center}
		\textbf{CONTINUE YOUR ANSWER FOR #1 BELOW THIS LINE } \\ 
		\hrulefill
		\pagebreak
	\end{center}
}

% Some  environments for having students showing their work
\newcommand{\showyourwork}[1]{
	\afterquestionvskip
	\begin{center}
		\textbf{SHOW YOUR WORK FOR #1 BELOW THIS LINE} \\ 
		\hrulefill
		\pagebreak 
	\end{center}
}

\newcommand{\showyourworkcontd}[1]{
	\afterquestionvskip
	\begin{center}
		\textbf{CONTINUE SHOWING YOUR WORK FOR #1 BELOW THIS LINE} \\ 
		\hrulefill
		\pagebreak 
	\end{center}
}

\newcommand{\showyourworkopt}[1]{
	\afterquestionvskip
	\begin{center}
		\textbf{OPTIONALLY, SHOW YOUR WORK FOR #1 BELOW THIS LINE} \\ 
		\hrulefill
		\pagebreak 
	\end{center}
}

\newcommand{\showyourworkoptcontd}[1]{
	\afterquestionvskip
	\begin{center}
		\textbf{CONTINUE OPTIONALLY SHOWING YOUR WORK FOR #1 BELOW THIS LINE} \\ 
		\hrulefill
		\pagebreak 
	\end{center}
}


\newcommand{\freespace}[1]{
	\afterquestionvskip
	\begin{center}
		\large \textbf{SCRAP SPACE BELOW} \\ 
		\hrulefill
		\pagebreak 
	\end{center}
}

\newcommand{\notespage}{
	\pagebreak
	\begin{center}
		\Large \textbf{SCRAP PAPER} \\ \hrulefill
	\end{center}
	\pagebreak
}

\newcommand{\indbase}[1]{
	\begin{center}
		\textbf{WRITE YOUR INDUCTIVE BASE BELOW THIS LINE} \\
		\hrulefill
	\end{center}
   	\vspace{#1}
}

\newcommand{\indhypothesis}[1]{
	\begin{center}
		\textbf{WRITE YOUR INDUCTIVE HYPOTHESIS BELOW THIS LINE} \\
		\hrulefill
	\end{center}
   	\vspace{#1}
}


\newcommand{\indstep}[1]{
	\begin{center}
		\textbf{WRITE YOUR INDUCTIVE STEP BELOW THIS LINE} \\
		\hrulefill
	\end{center}
   	\vspace{#1}
}


\newcommand{\contindbase}[1]{
	\begin{center}
		\textbf{CONTINUE YOUR INDUCTIVE BASE BELOW THIS LINE} \\
		\hrulefill
	\end{center}
   	\vspace{#1}
}

\newcommand{\contindhypothesis}[1]{
	\begin{center}
		\textbf{CONTINUE YOUR INDUCTIVE HYPOTHESIS BELOW THIS LINE} \\
		\hrulefill
	\end{center}
   	\vspace{#1}
}


\newcommand{\contindstep}[1]{
	\begin{center}
		\textbf{CONTINUE YOUR INDUCTIVE STEP BELOW THIS LINE} \\
		\hrulefill
	\end{center}
   	\vspace{#1}
}

\newcommand{\standardinductionspace}{
	\indbase{2.5in}
	\indhypothesis{1.5in}
	\pagebreak
	\indstep{\paperheight}
}

% Space for T/F:
\newcommand{\tfline}{\myline{.5cm}}

% For sets:
\newcommand{\curlybraces}[1]{\ensuremath{\lbrace #1 \rbrace}}

% For cardinalities:
\newcommand{\crd}[1]{\ensuremath{ \big \vert #1 \big \vert }}

\newcommand{\yesno}{{\footnotesize YES / NO}}
\newcommand{\truefalse}{{\normalsize  \bf TRUE / FALSE}}

% \item environments coupled with a line at the end, for students to write T and F in.
\newcommand{\tfitem}[1]{\item #1 \null\hfill \tfline}
\newcommand{\setitem}[1]{\tfitem{$\curlybraces{#1}$} }
% Emphasis
\newcommand{\F}{$\mathbf{F}$}
\newcommand{\T}{$\mathbf{T}$}
\newcommand{\False}{\textbf{False}}
\newcommand{\false}{\textbf{false}}
\newcommand{\True}{\textbf{True}}
\newcommand{\true}{\textbf{true}}
\newcommand{\TODO}{\textcolor{red}{\textbf{TODO}}}
\newcommand{\TBD}{\textcolor{red}{\textbf{TBD}}}
\newcommand{\codeemph}[1]{\texttt{\textbf{#1}}}
\newcommand{\Red}{\textcolor{red}{Red}}
\newcommand{\red}{\textcolor{red}{red}}
\newcommand{\makered}[1]{\textcolor{red}{#1}}
\newcommand{\Rbbst}{\textcolor{red}{Red}-black tree}
\newcommand{\rbbst}{\textcolor{red}{red}-black tree}
\usepackage{censor} 
\usepackage{textcomp} % for the cent symbol
\usepackage{multido}
\newcommand{\nullc}{\texttt{\textbackslash 0}}
\newcommand{\SPC}{\texttt{SPC}}
\usepackage{bbding} % Has a checkmark symbol reachable through \Checkmark
\newcommand{\checkmarkifsoln}{\ifshowsoln \textcolor{red}{\Checkmark} \fi}
\newcommand{\yesno}{{\footnotesize YES / NO}}
\newcommand{\truefalse}{{\normalsize  \bf TRUE / FALSE}}
\newcommand{\dialitem}[2]{\item[-] \textbf{#1}: ``\textit{#2}"} % For dialogue building.
% Repeating commands many times!
\newcommand{\Repeat}{\multido{\i=1+1}} 
\newcommand{\databox}[5]{
   \pagestyle{myheadings}
   \thispagestyle{plain}
   \newpage
   \setcounter{page}{1}
   \noindent
   \begin{center}
	   \framebox{
	      \vbox{
	        \vspace{2mm}
		    \hbox to 6.9in { {\bf #1 \hfill Sections: #2} }
	    	\vspace{8mm}
	        \hbox to 6.9in { {\huge \hfill   #3 \hfill } }
	        \vspace{4mm}
	       \hbox to 6.9in { { \Large \hfill  #4 \hfill } }
	        \vspace{8mm}
	        \hbox to 6.9in { {\it Student's first and last name: \myline{2.3in} \hfill Grade (grader only):\myline{.4in}} }
	        \vspace{2mm}
	        \hbox to 7.2in { {\it Student's Section Number : }\myline{1.0in} \hfill	
																			\begin{tabular}{|c|c|c|c|c|c|c|}\hline
																				\textbf{P1} & \textbf{P2} & \textbf{P3} & \textbf{P4} & \textbf{P5}   & \textbf{P6} & \textbf{P7} \\ \hline & & & & & & \\ \hline
			\end{tabular}
										
	        \vspace{2mm}}
	         \hbox to 1in {{\it Student's UID:} \myline{1.4in} }
   	        \vspace{1cm}
	        \begin{center}
	        	\textbf{University Honor Pledge:} \\[.2in]
	        	\emph{I pledge on my honor that I have not given or received \\
		        any unauthorized assistance on this assignment/examination.  }\\[.35in]		        
				\textbf{Print the text of the University Honor Pledge below}: \\[.2in]
				 \myline{5in} \\[.1in]
				\myline{5in} \\[.1in]
				\myline{5in} \\[.3in]
			\textbf{Signature:} \myline{2.8in}  \\
			\end{center}
	   	  } % vbox
	   } % framebox
   \end{center}
   \markboth{#4 -- #1}{#4 -- #1}
   \vspace*{4mm}
} % Edit this file for variable problem numbers in grading box
% Playing card support
\DeclareSymbolFont{extraup}{U}{zavm}{m}{n}
\DeclareMathSymbol{\varheart}{\mathalpha}{extraup}{86}
\DeclareMathSymbol{\vardiamond}{\mathalpha}{extraup}{87}
\newcommand{\hrt}{\textcolor{red}{$\varheart$}}
\newcommand{\dmd}{\textcolor{red}{$\vardiamond$}}
\newcommand{\spd}{$\spadesuit$}
\newcommand{\clb}{$\clubsuit$}
%%%%%%%%%%%%%%%%%%%%%%%%%%%%%%%%%%%%%%%%%%%%%

% Tweak the following based on what you think the current document needs:
\usepackage[inner=1.5cm,outer=1.5cm,top=2cm,bottom=1.5cm]{geometry}
\setlength{\parindent}{2em}
\setlength{\itemindent}{.5in}

% Title of the current document
\title{CMSC250, Fall '17: Final Exam}

\begin{document}

% Box at the top of every first page.
\databox{CMSC 250, Fall 2017}{all}{Final Exam}{Friday, 12-15-2017} 

\vspace{-.2in}

\section*{Exam Guidelines / Rules / Assumed Facts}
\label{sec:guidelines}
 
\begin{itemize}
	\item \textbf{Turn off all unapproved electronic devices} (e.g phones, tablets, laptops, but not calculators if you feel you need them). Setting a device on ``silent" or ``sleep" mode does not constitute it being turned \textbf{off}: Your device is turned off if and only if it requires pushing a power button to begin the execution of a bootloader. \textbf{Proctors reserve the right to confiscate an electronic device if it is not turned off according to the definition above.}
	\item Write \textbf{neatly}. If we can't read your response, you will receive no credit for it. Use the scrap paper provided at the end of the exam for note-taking. You can ask for extra scrap paper if you need it.
	\item There are \textbf{6 (six)} problems in this exam, with a total grade value that adds to $\mathbf{100}$ \textbf{(one hundred)}, as well as an extra credit problem, worth $\mathbf 1$ {\bf (one)} point.
%as well as \textbf{an additional extra credit problem}, worth \textbf{1 (one)} point.
	\item The exam  has been printed \textbf{two-sided}, \textbf{stapled on the top-left corner} and spans $\mathbf{18}$  (\textbf{eighteen}) pages across $\mathbf{9}$ (\textbf{nine}) sheets. The last $\mathbf{2}$ {\bf (two)}  pages are \textbf{scrap paper} for your notes. 
	\item The total time allocated for this exam is \textbf{115 (one hundred and fifteen)} minutes.
	\item During the \textbf{last 5 (five) minutes} of the exam, you may \textbf{not} leave the classroom (e.g to go to the bathroom, or because you're done). 
	
\end{itemize}

\pagebreak

\section*{Provided materials}

Table \ref{tbl:propLogicAxioms} contains a number of logical equivalences that we have discussed in class.

\begin{table}[H]
	\centering
	\begin{tabular}{|p{2.8in} | c | c |} \hline
		\textbf{Commutativity of binary operators} & $p \land q \equiv q \land p$ & $p \lor q \equiv q \lor p$ \\ \hline
	\textbf{Associativity of binary operators} & $(p \land q) \land r \equiv p \land (q \land r)$ &  $(p \lor q) \lor r \equiv p \lor (q \lor r)$ \\ \hline
	\textbf{Distributivity of binary operators} & $p \land (q \lor r) \equiv (p \land q) \lor (p \land r)$ & $p \lor (q \land r) \equiv (p \lor q) \land (p \lor r)$ \\ \hline
	\textbf{Identity laws} & $p \land t \equiv p$ & $p \lor c \equiv p$ \\ \hline
	\textbf{Negation laws} & $p \lor \neg p \equiv t$ & $p \land \neg p \equiv c$ \\ \hline
	\textbf{Double negation} & \multicolumn{2}{c|}{$\neg (\neg p) \equiv p$}  \\ \hline 
	\textbf{Idempotence} & $p \land p \equiv p$ & $p \lor p \equiv p$ \\ \hline
	\textbf{De Morgan's axioms} & $\neg (p \land q) \equiv \neg p \lor \neg q$ & $\neg (p \lor q) \equiv \neg p \land \neg q$\\ \hline
	\textbf{Universal bound laws} & $p \lor t \equiv t$ & $p \land c \equiv c$ \\ \hline
	\textbf{Absorption laws} & $p \lor (p \land q) \equiv p$ & $p \land (p \lor q) \equiv p$ \\ \hline
	\textbf{Negations of contradictions / tautologies} & $\neg c \equiv t$ & $\neg t \equiv c$ \\ \hline
	\textbf{Equivalence between biconditional and implication} & \multicolumn{2}{c|}{$a \Leftrightarrow b \equiv (a \Rightarrow b) \land (b \Rightarrow a)$} \\ \hline
	\textbf{Equivalence between implication and disjunction} & \multicolumn{2}{c|}{$a \Rightarrow b \equiv \neg a \lor b$} \\ \hline
	\end{tabular} \vspace{.1in}
	\caption{A number of propositional logic axioms you can refer to.}
	\label{tbl:propLogicAxioms}
\end{table}

Table \ref{tbl:propLogicRules} contains a number of rules of natural deduction / inference that we have discussed in class.

\begin{table}[H]
	\centering
	\begin{tabular}{|p{2.8cm}|p{2.8cm}|p{2.8cm}|p{2.8cm}| p{2.8cm}|}\hline
		\textbf{Modus Ponens} & \textbf{Modus Tollens} & \textbf{Disjunctive addition} & \textbf{Conjunctive addition} & \textbf{Division into Cases}\\ \hline
			$\begin{aligned}
		p  \\
		p \Rightarrow s \\
		\therefore s
	\end{aligned}$ & 
	 $\begin{aligned}
		\neg s  \\
		p \Rightarrow s \\
		\therefore \neg p
	\end{aligned}$ & 
	 $\begin{aligned}
		p  \\
		\therefore p \lor s
	\end{aligned}$ &
	$\begin{aligned}
		p, s \\
		\therefore p \land s
	\end{aligned}$ & 
	$\begin{aligned}
		p \lor s \\
 		p \Rightarrow r \\ 
		s \Rightarrow r \\
 \therefore r\\
	\end{aligned}$ \\ \hline
		 
		 \textbf{Conjunctive Simplification} & \textbf{Disjunctive syllogism} & \textbf{Hypothetical syllogism} &  \textbf{Resolution}  & \textbf{A bunny with a pancake on its head} \\ \hline
		 $\begin{aligned}
		p \land s \\
		\therefore p, s
	\end{aligned}$ 	& 
		$\begin{aligned}
		p \lor s \\
		\neg p \\
		\therefore s
	\end{aligned}$ &
	$\begin{aligned}
		p  \Rightarrow s \\
		s \Rightarrow r \\
		\therefore p \Rightarrow r
	\end{aligned}$ &  
	$\begin{aligned}
	p \lor s\\
	(\neg s )\lor z\\
	\therefore p \lor z
	\end{aligned}$ & \vspace{.1cm}
	\includegraphics[scale=.25]{img/pancake_bunny}\\ \hline
	\end{tabular} 
	\caption{Propositional Logic rules of inference / natural deduction you may refer to.}
	\label{tbl:propLogicRules}
\end{table} 

You may use the following mathematical facts {\bf without proof}:
{\large 
\begin{itemize}
	\item  $\Z$ is closed under addition, subtraction and multiplication.
	\item $\N$ is closed under addition and multiplication.
	\item  $0 \in \N$.
	\item $\sqrt2$ is an irrational number.
\end{itemize}
}

\pagebreak

Finally, table \ref{tbl:setTheoryAxioms} contains various Set Theory axioms that you can refer to.

\begin{table}[H]
	\centering
\begin{tabular}{| p{1.8in} | c | c |} \hline
	Commutativity  & $A \cup B = B \cup A$ & $A \cap B = B \cap A$ \\ \hline
		Associativity of union \& intersection & $(A \cap B) \cap C = A \cap (B \cap C)$ &  $(A \cup B) \cup C = A \cup (B \cup C)$ \\ \hline
		Distributivity of union \& intersection & $A \cap (B \cup C) = (A \cap B) \cup (A \cap C)$ & $A \cup (B \cap C) = (A \cup B) \cap (A \cup C)$ \\ \hline
		Identity laws & $A \cap U = A$ & $A \cup \emptyset = A$ \\ \hline
		Inverse laws & $A \cup  A^c = U$ & $A \cap A^c = \emptyset$ \\ \hline
Double complementation &\multicolumn{2}{c|}{ $  (A^c)^c = A$}  \\ \hline  
		Idempotence & $A \cap A = A$ & $A \cup A = A$ \\ \hline
		De Morgan's axioms & $  (A \cap B)^c =   A^c \cup   B^c$ & $  (A \cup B)^c =   A^c \cap B^c$\\ \hline
		Universal bound (Domination) laws & $A \cup U = U$ & $A \cap \emptyset = \emptyset$ \\ \hline
		Absorption laws & $A \cup (A \cap B) = A$ & $A \cap (A \cup B) = A$ \\ \hline
	   Absolute Complements of empty set / domain & $  \emptyset^c = U$ & $  U^c = \emptyset$ \\ \hline
	   Relationship between relative and absolute complement & \multicolumn{2}{c|}{$A - B = A \cap B^c$ } \\ \hline 
	\end{tabular} 
	\caption{Various Set Theory axioms you may refer to.}
	\label{tbl:setTheoryAxioms}
\end{table}


\vspace{2in}
\frontrowcenter{The exam problems begin on the next page.}
\pagebreak


\problemhdr{1: Logic  {\em (15 pts)}}

\questionhdr{(a): Truth tables (7 pts)}

{\large  Complete the  following \textbf{truth table} for the logical expression {\Large $(p \lor q) \land (\neg (z \lor q) )$ }. To start you off, we are giving you the first three columns. You can draw and fill as {\bf many columns as you need}, but please be careful to \textbf{not run out of space}. Write {\bf neatly}; if we can't make the difference between a \T{} and an \F{}, we will be forced to take off points! } 
% \answerspace{QUESTION (a)}

\begin{center}
	\begin{table}[H] 
		\Large 
		\setlength{\tabcolsep}{16pt}
		\renewcommand{\arraystretch}{1.5}
		\begin{tabular}{|c|c|c|p{5.25in}|} \hline 
			$p$ & $q$ & $z$ & \\ \hline
			\F & \F & \F & \\ \hline 		
			\F & \F & \T & \\ \hline 		
			\F & \T & \F & \\ \hline 		
			\F & \T & \T & \\ \hline
			\T & \F & \F & \\ \hline 		
			\T & \F & \T & \\ \hline
			\T & \T & \F & \\ \hline 		
			\T & \T & \T & \\ \hline												
		\end{tabular}
	\end{table}
	\vspace{.3in} 
	\begin{table}[H] 
		\Large 
	\renewcommand{\arraystretch}{1.5}
		\begin{tabular}{|p{\textwidth}|} \hline    \\ \hline	\\ \hline \\ \hline \\ \hline \\ \hline \\ \hline \\ \hline \\ \hline \\ \hline
		\end{tabular}
	\end{table}
\end{center}
\pagebreak
\questionhdr{(b): Negating quantifiers (8 pts)} \vspace{-.1in}

{\large In the following quantified expressions, push the negation operator ($\neg$) {\bf as far inside the expression as possible.} You do {\bf not} need to show us your steps; if you prefer working in your scrap paper and just showing us your final answer, that is fine; if you prefer to show us all the steps below, that is {\bf also} fine, but \textbf{be careful not to run out of space}. In addition, you do {\bf not} need to care about the truth values of the statements; that is, it is \textbf{not important} whether the statements themselves or their negations are \True{} or \False{}; just push the negation operator as discussed. 

\begin{enumerate}[label=(\roman*)]  
{\Large 
  \setlength\itemsep{.3em}	\item $\neg \big (\exists q \in \Q \big ) \big [ q^{2} \in \Q \big ]$  {\em (1 pt)}
	\item $\neg \big (\forall a \in \Z ) (\exists b \in \Z) \big [ (a + b) \notin \Z \big ]$  {\em (3 pts)}
	\item $\neg \big (\forall r \in \R \big ) \big [ \big ( (r > 0) \land ( \sqrt{r} \in \N)\big ) \Rightarrow (r \in \N) \big ]$  {\em (4 pts)} }
\end{enumerate}
} \vspace{-.4in}
\answerspace{QUESTION (b)}

\problemhdr{2: Various {\em (2 pts each for 20 pts total)}}

{\large For all of the following statements,  \textbf{circle} \True{} or \False{} depending on whether you believe the statement to be true or false respectively. You do {\bf not} need to justify your answers. Recall that  {\Large $S^c$} is the {\em universal complement} of set {\Large $S$}.  Note that for the questions that involve sets $A, B$, we are asking you to tell us if you believe that the statement is \True{} or \False{} for {\bf all possible sets} $ A$, $ B$.
} 
\begin{center}
	\setlength{\tabcolsep}{40pt}
	\begin{table}[H]
		\centering 
		\large
		\begin{tabular}{|cc|}    \hline & \\ 
				{\bf \Large Statement} & {\bf  \Large True or False?} \\ & \\ \hline   & \\ 
	 					 {\Large $A \cup \emptyset = A$	} & \truefalse  \\ & \\ \hline & \\
						 {\Large $A \cap A^c \subseteq \emptyset$ 	}& \truefalse \\ & \\ \hline & \\
						{\Large	$A - B = A \cup B^c$ }& \truefalse \\ & \\ \hline & \\ 	
				{\Large		  $\lbrace \emptyset \rbrace \cup \lbrace \rbrace  = \lbrace \rbrace$ }& \truefalse \\ & \\ \hline & \\ 
									{\Large	$\lbrace \lbrace \emptyset \rbrace \rbrace \subseteq \lbrace \lbrace \emptyset, \lbrace \emptyset \rbrace \rbrace \rbrace$ }& \truefalse \\ & \\ \hline & \\ 
					 The function {\Large $f:\R  \mapsto \R$} defined as: & \ \\ & \\ & \\ {\Large $f(x)=x^3$} & \truefalse  \\ & \\ & \\  is a {\bf bijection}. & \\ & \\ \hline & \\ 
					 					 The function {\Large $f:\Z \mapsto \R$ } defined as: &  \\ & \\ & \\{\Large $f(x) = \begin{cases} x^2, & x < 0 \\ -x^2, & x \geq 0 \end{cases}\quad $ }& \truefalse  \\ & \\ & \\  is a {\bf bijection}. & \\ & \\ \hline 
		\end{tabular}
	\end{table}
\end{center}	

\pagebreak

\begin{center}
	\setlength{\tabcolsep}{40pt}
	\begin{table}[H]
		\centering 
		\large
		\begin{tabular}{|cc|}  \hline & \\ 

					The relation  & \\ &  \\ & \\ {\Large $R = \{ (x, y) \in \R \times \R \mid  (x - y)^2 \leq 4 \}$ }& \truefalse  \\ & \\ & \\  is {\em reflexive}, {\em symmetric} {\bf and} {\em transitive}. & \\ & \\ \hline & \\ 
				The set of positive rationals $\Q^{>0}$   & \\ & \\ 
					has the same \textbf{cardinality} as & \truefalse \\ & \\
					the set of integers $\Z$. &  \\ & \\ \hline	& \\ 				 
					The set {\Large $S = \big \{ f \mid f: \N \mapsto \{0, 1\} \big \} $}& \\ & \\
					(set of \textbf{all possible functions} from $\N$ \textbf{only to the} & \truefalse  \\ 					&  \\ 
					 \textbf{two integers} $\mathbf 0$, $ \mathbf 1$) is {\bf un}countable. & \\ & \\ \hline 
		\end{tabular}
	\end{table}
\end{center}	

%%%%% Enumerate-based approach  %%%%%%%%%
\begin{comment}

	{\Large 
	
	\begin{enumerate}[label=(\alph*)]
		\item $A \cup \emptyset = A$ \tfline
		\item $A \cap A^c \subseteq \emptyset$\tfline
		\item $\lbrace \emptyset \rbrace \cup \lbrace \rbrace  = \lbrace \rbrace$ \tfline
		\item $A - B = A \cup B^c$ \tfline
		\item $A^c - A = \emptyset$ \tfline
		\item $(B - C)^c = B^c - C^c$ \tfline
		\item $\lbrace \lbrace \emptyset \rbrace \rbrace \subseteq \lbrace \lbrace \emptyset, \lbrace \emptyset \rbrace \rbrace \rbrace$ \tfline
		\item The function $f:\Z \mapsto \R$ defined as $f(x)=x^3$ is a bijection. \tfline
		\item The function $f: \R \mapsto \R$ defined as $f(x) = \begin{cases} x^2, & x < 0 \\ -x^2, & x \geq 0 \end{cases}\quad $ is a bijection. \tfline
		\item The relation $R = \{ (x, y) \in \R \times \R \mid  (x - y)^2 \leq 4 \}$ is {\em reflexive}, {\em symmetric} {\bf and} {\em transitive}. \tfline
	\end{enumerate} \vspace{-.4in}

} 

\end{comment}
\vspace{-.65in}
\freespace{PROBLEM 2}

\problemhdr{3: Formal proofs / Number Theory {\em (15 pts)}}

{\large Using {\bf either} a \textbf{direct} {\bf or} an {\bf indirect} proof methodology, prove that the following statements are \True. 
\begin{enumerate}[label=(\alph*)]
	\item For any integers {\Large $a,b,c$} with {\Large $a \neq 0$}, if {\Large $(a \vert b)$} \textbf{and} {\Large $(a \vert c)$}, then {\Large $(\forall s, r \in \mathbb{Z})[\ a\ \vert\ (s\cdot b +r \cdot c)\ ]$.} {\em (7 pts)}
	\item  {\Large $\sqrt 2 + 1$} is {\bf not} divisible by {\Large  $3$ }. You may use the fact  that {\Large $\sqrt2 \notin \Q$} {\bf without proof.} \em (8 pts)
\end{enumerate}
} \vspace{-.4in}

\answerspace{PROBLEM 3}
\additionalanswerspace{PROBLEM 3}

\problemhdr{4: Induction {\em (20 pts)}}

{\large Let $a_n$ be a sequence recursively defined as follows: }

{\Large

\[ a_n = \begin{cases}3, & n = 0 \\ 4, & n = 1 \\ a_{\scriptsize n-1} + a_{\normalsize n-2}\ \   - \ n , & n \geq 2 \end{cases}\]		
}

{\large Note that, in the recursive case of $a_n$, the quantity $-n$ is {\bf not} an index into a sequence term, like $n-1$ and $n-2$ are! Another way to write the recursive case would be: {\Large \[ -n + a_{n-1} + a_{n-2}\] } }

{\large Using {\bf whichever} mathematical induction principle you consider appropriate, so long as you {\bf sharply delineate} the {\bf Inductive Base}, {\bf Hypothesis} and {\bf Step}, prove that}

{\Large \[ (\forall n \in \N)[a_n = n + 3]\]}

 {\large Please write {\bf neatly} in the Inductive Step. If we cannot understand what you are writing, we will {\bf not} be able to give you \textbf{any credit} for the Step!}
 \vspace{-.2in}
 
\answerspace{PROBLEM 4}
{\additionalanswerspace{PROBLEM 4}

\problemhdr{5: Combinatorics  {\em (25 pts)}}

{\large Answer the following questions on the small line available to you after each and every one of them. You do {\bf not} need to justify your answers. Remember that in combinatorics problems, we do {\bf not} want you to perform messy calculations: Just leave your result in terms of permutation or combination symbols, factorials, powers, sums or products of all of those, etc.

\vspace{.1in} 
\begin{enumerate}[label=(\alph*)]
%		  \setlength\itemsep{1em}
		\item \doublespacing How many different strings of length $\mathbf 5$, $\mathbf 6$, or $\mathbf 7$ ({\bf five, six} or \textbf{seven}) \\ can we create with the English alphabet? Note that the English \\ alphabet has $\mathbf{26}$ (\textbf{twenty-six}) characters and, of course, characters \\ can be \textbf{repeated}.  For example, there are two `e's in the word ``repeat". \null\hfill \myline{1in} {(\em 1 pt)}
	\item Calculate the (distinguishable) permutations of the following strings:
		\begin{enumerate}[label=(\roman*)]
				  \setlength\itemsep{0.6em}
			\item {\em iowa } \null\hfill \myline{1in} {\em(1 pt)}
			\item {\em oregon} \null\hfill \myline{1in} {\em(2 pts)}
			\item {\em mississippi} \null\hfill \myline{1in} {\em (3 pts)}
		\end{enumerate}
		\item A bit-string is a string that contains only $\mathbf{1}$s and $\mathbf{0}$s ({\bf one}s and \\ {\bf zero}es).  How many bit-strings of length $\mathbf{12}$ (\textbf{twelve}) have...
	\begin{enumerate}[label=(\roman*)]  \setlength\itemsep{0.6em}
	  	\item \textbf{Exactly} 5 (five) 1s? \null\hfill \myline{1in} {\em (1 pt)}
		\item \textbf{At most} 5 (five) 1s? \null\hfill \myline{0.9in} { \em (2 pts)}
		\item \textbf{At least} 5 (five) 1s? \null\hfill \myline{1in} {\em (2 pts)}
		\item An \textbf{equal number} of 1s and 0s? \null\hfill \myline{1in} {\em (2 pts)}
	\end{enumerate} \singlespacing
	\item  Suppose that we have 50 (fifty) students in a classroom. 
	\begin{enumerate}[label=(\roman*)] 
		\item  \doublespacing \setlength\itemsep{1.5em}  We want to take a photo of the students. In how many ways  \\ can we order them in a single row? \null\hfill \myline{1in} {\em(1 pt)} 
		\item  If we want $11$ (eleven) of these students to  sit in a row of $11$ \\  (eleven) seats, in how many ways can we do that? Recall that \\  when 2 (two) or more people are sitting in a row, their {\bf order} \\ 
matters.  \null\hfill \myline{1in}  {\em (2 pts)}
		\item  Suppose that $\mathbf{20}$ (\textbf{twenty}) of those students have a first name \\  that begins with a `K'. $\mathbf{10}$ (\textbf{ten}) students have a last name \\ that begins with an `M'.  $\mathbf 5$ (\textbf{five}) students have a first name \\ that starts with a `K' {\bf and} a last name that begins with an `M'. 
		\vspace{1em}
		\begin{enumerate}[label=(\roman*)]
			\item How many students have  {\bf neither} a first name  that \\ begins with a `K' {\bf nor} a last name that begins with  an `M'? \null\hfill \myline{1in} {\em (2 pts)} 
			\item How many different {\bf pairs} of students have {\bf neither} \\  first name  that  begins with a `K' {\bf nor} a last name \\ that begins with an `M'? \null\hfill \myline{1in} {\em (1 pt)} 
	\end{enumerate}	
		\item Suppose that a university department contains $\mathbf{10}$ (\textbf{ten}) men \\ and $\mathbf{15}$ (\textbf{fifteen}) women. How many ways are there to form a  \\ committee with $\mathbf{6}$ (\textbf{six}) members if: \vspace{1em}
		\begin{enumerate}[label=(\roman*)] \setlength\itemsep{1.5em} 
			\item It must have \emph{the same} number of men and women? \null\hfill \myline{1in} {(\em 1 pt)}
			\item \doublespacing It must have \emph{more} women than men?  Please note that \\ $\mathbf 0$ \textbf{(zero)} men and a  \textbf{non-zero} number of women \textbf{still} \\ means that there are \textbf{more women than men} in the \\ committee!   \null\hfill \myline{1in}  {\em (2 pts)}
		\end{enumerate}		
	\item  \doublespacing Suppose that we have a basket with 4 (four) different kinds \\  of apples: 8 (eight) Braeburn, 2 (two) Fuji, 4 (four) Elstar \\ and 6 (six) Gala. We {\bf randomly} pick 4 (four) apples from \\ the basket. What is the {\bf probability} that the apples picked \\ will all be of  \textbf{a different} \textbf{kind}? \null\hfill \myline{1in}  {(\em 2 pts)}
\end{enumerate} 
		\end{enumerate}
 \vspace{-.5in}
 
 \pagebreak

\problemhdr{6: Show me what you got {\em (5 pts)}}

{\large Suppose that $n$ and $m$ are natural numbers. Using {\bf whichever} mathematical induction principle you consider appropriate, so long as you {\bf sharply delineate} the {\bf Inductive Base}, {\bf Hypothesis} and {\bf Step}, prove that  }

{\Large $$\Sum_{i=0}^{n}  \binom{m + i - 1}{i} = \binom{m + n}{n}$$ } 

{\large 
Hints: 
\begin{itemize} 
	\item Your induction should be on $\mathbf n$.
	\item When solving the inductive step:
	\begin{itemize}
		\item  \textbf{Keep in mind the answer that you want}.  This will give you a clue on how to manipulate the expression to reach it.
		\item If you want to use combinatorial identities that we have talked about in class, \textbf{you are absolutely allowed to do so}. However, please make sure you \textbf{explicitly state  which identity you are using} so that we can understand what you are doing during grading.
\end{itemize}	
	
\end{itemize}

 {\large Please write {\bf neatly} in the Inductive Step. If we cannot understand what you are writing, we will {\bf not} be able to give you \textbf{any credit} for the Step!}

}
\vspace{-.3in}
\answerspace{PROBLEM 6}
\additionalanswerspace{PROBLEM 6}

\problemhdr{7:  General Trivia {\em (1 pt extra credit)}}

{\large In the legendary real-time-strategy video game {\em Age of Empires II: The Age of Kings} (as well as \\ \ \\  its various expansion packs), what does the cheat code {\tt howdoyouturnthison} accompish?}  \myline{.87 in} \\ \ \\ \myline{\textwidth}

\pagebreak

 
\Repeat{2}{\notespage}
\end{document}